\documentclass[11pt, oneside]{article}
\usepackage{geometry}
\geometry{letterpaper}                   		% ... or a4paper or a5paper or ... 
%\geometry{landscape}                		% Activate for rotated page geometry
%\usepackage[parfill]{parskip}    		% Activate to begin paragraphs with an empty line rather than an indent
\usepackage{graphicx}				% Use pdf, png, jpg, or eps§ with pdflatex; use eps in DVI mode
								% TeX will automatically convert eps --> pdf in pdflatex		
\usepackage{amssymb}
\usepackage{amsmath}
\usepackage[utf8]{inputenc}
\usepackage[spanish]{babel}

% Set Environments
\newtheorem{definition}{Definición}

%SetFonts

\newcommand{\eu}{\ensuremath{\mathrm{e}}}
\newcommand{\du}{\ensuremath{\mathrm{d}}}

%SetFonts


\title{Cooperación Internacional de efectos económicos de la propagación de una enfermedad}
\author{Mario A. García-Meza\\ César Gurrola-Ríos \\ Julieta E. Sánchez-Cano}
%\date{}							% Activate to display a given date or no date

\begin{document}
\maketitle

\abstract{Se realiza una aplicación de teoría de juegos diferenciales cooperativos para la determinación de los parámetros de control que una cooperación entre naciones debe de aplicar en gasto público de salud ante la aparición de un problema conjunto.}

\section{Introducción}

% What do we do and why is it relevant
The role of a government in situations where externalities are present has become the subject of intense discussion in light of the COVID-19 outbreak in 2020. There are at least two big areas where a government is accountable when an event of such magnitude occurs: health policy and economics. Unfortunately, there seems to be a tradeoff to be considered: closing down the economy for the sake of health immediately hurts the economy.

Even though there is evidence that such tradeoff does not exist \cite{Correia}, the immediate consequences do not make that evident to most policymakers. Further, there might be political incentives for playing down the effects of an outbreak in political discourse and keep the economy open.

In this paper we explore the incentives given by this tradeoff and the policy decisions that have to be made by agents that need to cooperate in order to minimize the health and economics costs given by an exogenous virus outbreak. We consider as an example the case of federalism and states within a country with variable policies. We can find such a case in the United States, with a general mandate but differing policies. More importantly, there is free mobility within states that require some degree of cooperation within states. Such scenario is also present in the European Union, where each country can adopt their individual policies, but free mobility cannot be easily overriden in order to enforce such policies (cf. \cite{EC-COVID}).

Naturally, the ideas of the model apply to other countries and regions that face the same situation, where a cooperative set of rules is needed. For example, Mexico adopted a nationwide policy at the beginning of the outbreak, but by the end of May announced that the health policies should be implemented by states \cite{SSA-NN}. As we will see, the nature of the incentives at hand make this decision less than optimal and may be improved by assuming a nationwide mandate.

Our approach involves a cooperative differential game with $n$ agents. An agent in this model is a decision maker, which can be thought of as a state or local government. Each of these agents have to decide its optimal policy choice in order to minimize negative outcomes. A discussion about the economics of pandemics, public policy and the tradeoffs to be considered is presented in section \ref{Policy} while the details of our model are described in section \ref{model}.

The computation of the solution is presented in section \ref{IDP}, along with a detailed description of the method in which we arrive at the conclusions presented. Section \ref{Conclude} provide a discussion of the insights that our analysis yield and how they can be informative for policymakers that face the kind of incentives described by this article.


\section{The policy implications of a virus outbreak}\label{Policy}
% Facemasks
Since the 1918 Spanish Flu, we as humanity have not face a similar challenge as the one that COVID-19 poses. A series of policy questions arise under these circumstances: should face masks be mandatory? how long and how strict should a lockdown be?


\section{The model} \label{model}


\section{Stability of cooperation}\label{IDP}

\section{Concluding remarks} \label{Conclude}

\section{Formulación de un juego diferencial}

Considere un juego diferencial de $N$ países $\Gamma(t_0,x_0)$ comenzando en el momento $t_0$ en el tiempo con un estado inicial constante $x_0$ y una duración de $T-t_0$. Aquí, $T$ es una variable aleatoria con función de distribución acumulada (FDA) $F(t)$, donde $t\in[t_0,\infty)$ es el momento del tiempo en el que el juego termina. La FDA se asume ser una función contínua no decreciente que satisface

\begin{enumerate}
	\item $F(t_0) = 0$,
	\item $\lim_{t\to\infty}F(t)=1$
\end{enumerate}

Adicionalmente, existe una función contínua casi seguramente $f(t) = F'(t)$, llamada la función de densidad tal que

\begin{equation}
	F(t) = \int_{t_0}^t f(\tau)d\tau \qquad t \in [t_0,\infty]
\end{equation}

Permitamos que la dinámica del sistema sea descrita por la siguiente ecuación diferencial ordinaria:
\begin{equation}
	\dot{x} = g(x, u_1,\dots,u_N), \quad x\in\mathbb{R}^m, u_i \in U\in \mbox{comp}(\mathbb{R}), x(t_0) = x_0
\end{equation}
donde $g : \mathbb{R}^m\times \mathbb{R}^N \to \mathbb{R}^m$ es una función valuada en vector que satisface los requisitos estándar de existencia y unicidad (cite).

El pago instantáneo del jugador $i$ en el momento $\tau$, $\tau \in [t_0,\infty)$ se ha definido como $h_i(x(\tau), u_i(\tau))$. Entonces, el pago esperado del jugador $i$, donde $i = 1,\dots,N$ se evalúa por medio de la igualdad

\begin{equation}
	K_i(t_0, x, u) = \int_{t_0}^\infty\int_{t_0}^t h_i(x(\tau), u_i(\tau)) d\tau dF(t) = \int_{t_0}^\infty\int_{t_0}^t h_i(x(\tau), u_i(\tau)) d\tau f(t) dt
\end{equation}

Se define la estrategia óptima en el sentido de Pareto en el juego $\Gamma(t_0,x_0)$ como la $n$-tupla de controles $u^*(t) = (u_1^*, \dots, u_n(t))$ que maximiza el pago esperado conjunto de los jugadores:

\begin{equation}
	(u_1^*(t)\dots,u_n^*(t)) = {\arg\max}_u \sum_{i=1}^nK_i(t_0,x,u).
\end{equation}

Por lo tanto, la solución óptima en sentido de Pareto del juego $\Gamma(t_0,x_0)$ es $(x^*(t), u^*(t))$ y el pago total óptimo $V(x_0)$ es

\begin{equation}
	V(x_0, t_0) = \sum_{i=1}^n K_i(t_0,x^*,u^*) = \sum_{i=1}^n \int_{t_0}^\infty \int_{t_0}^t h_i(x^*(\tau), u_i^*(\tau))d\tau f(t)dt.
\end{equation}

El juego se divide en subjuegos a lo largo de la trayectoria. Para el conjunto de subjuegos $\Gamma(\vartheta, x^*(\vartheta))$, con $\vartheta > t_0$, que ocurre en la trayectoria óptima $x^*(\vartheta)$ es posible también definir el pago integral total esperado 

\begin{equation}
	V(x^*(\vartheta), \vartheta) = \sum_{i=1}^n \int_{\vartheta}^\infty \int_{\vartheta}^t h_i(x^*(\tau), u_i^*(\tau))d\tau dF_\vartheta(t),
\end{equation}

donde $F_\vartheta(t)$ es una función de distribución acumulada condicional definida por

\begin{equation}
	F_\vartheta(t) = \frac{F(t) - F(\vartheta)}{1 - F(\vartheta)}, \qquad t \in [\vartheta, \infty),
\end{equation}
con su respectiva función de densidad condicional

\begin{equation}
	f_\vartheta(t) = \frac{f(t)}{1 - F(\vartheta)}, \qquad t \in [\vartheta, \infty)
\end{equation}


\section{Los efectos de una pandemia en la economía}

Lorem ipsum

\section{El modelo} 

Consideramos un modelo donde las infecciones por COVID-19 tienen un efecto a nivel mundial. Por simplicidad, asumimos que la tasa de infectados es una variable contínua determinista. Sea $I$ el conjunto de regiones involucradas en el juego de la reducción de externalidades por contagio del virus. Denote el nivel de actividad económica de la región $i (i = 1,\dots,n)$ con $e_i$, donde la actividad económica es un determinante indirecto de la tasa de contagios de la comunidad general. En particular, denote con $S(t)$ al \emph{stock} de personas infectadas con el virus en el momento $t$. Entonces el número de infectados se puede describir por medio de la ecuación diferencial 

\begin{equation}
	\frac{d S(t)}{dt} = \dot{S}(t) = \sum_{i \in I} e_i(t) - r S(t), \qquad S(0) = S_0,
\end{equation}

donde $r$ es la tasa de recuperación de los pacientes. Asumimos por simplicidad que esta tasa es la misma en todos los países, aunque existe evidencia de que esta depende mucho de factores como la infraestructura hospitalaria y la capacidad de respuesta del sistema de salud. 

Todos los países buscan minimizar la suma de los costos tanto de los esfuerzos de contención por distanciamiento social como los daños económicos y de pérdidas humanas que conlleva la enfermedad. Denote por $C_(e_i)$ el costo de disminuir el nivel de actividad económica en $e_i$ debido a medidas como distanciamiento social, y $D_i(S)$ es el costo de los daños de la enfermedad. Note que el nivel de contagio mundial lleva a un nivel de costo diferente para cada país. Asumimos que ambas funciones son contínuamente diferenciables y convexas, con $C'(e_i) < 0$ y $D'(S) >$.

El problema de optimización del país $i$ está determinado por

\begin{equation}
	\min W_i(e,S) = \int_0^\infty \ensuremath{\mathrm{e}}^{\rho t}\{C_i(e_i) + D_i(S)\} dt.
\end{equation}

En otras palabras, cada país tiene como objetivo minimizar los costos que provienen de la reducción en la actividad económica (denotada por el vector $e = (e_1,\dots,e_n)$ y los costos económicos causados directamente por la enfermedad. Para efectos de simplicidad en la notación, se está dejando de lado el elemento del tiempo.

El costo denotado por $C_i(\cdot)$ incluye los costos derivados de la disminución de actividad económica debida a medidas de distanciamiento social y cuarentenas. Esto incluye la pérdida de ingresos y de consumo de personas cuyo trabajo no se puede realizar a distancia y que pierden su empleo de manera temporal o definitiva \cite{sdffsadaf}. También se incluye en esta medida el costo de oportunidad derivado de los cuidados en el hogar, producto de el cierre de las escuelas y centros de cuidado. Note que la convexidad de $C_i(\cdot)$ implica que el costo marginal de las medidas de distanciamiento social son mayores ante bajos niveles de distanciamiento, pero este disminuye para niveles mayores de distanciamiento. Existe evidencia que soporta este supuesto \cite{evidencia}.

La función $D_i(S)$ incluye los costos de salud pública implicados en el combate directo a la enfermedad, pero también puede incluir otras externalidades, cómo el costo agregado de salud en el futuro por los pacientes que no están siendo atendidos \cite{alguien}. También puede incluirse el costo de oportunidad de quienes no están trabajando debido a la enfermedad o al cuidado de personas con enfermedad \cite{}.
 
 
Naturalmente, si se tratara un problema de decisiones individuales sin externalidades, este sería un problema de optimización simple en el que se toma en cuenta la balanza de costos y beneficios para la toma de decisiones. Sin embargo, cuando el problema llega a convertirse en una pandemia, es necesario considerar las decisiones de los demás actores. En este caso, consideramos una metodología de teoría de juegos cooperativa. Esto implica que la solución que se obtenga de la misma debe de cumplir con ciertas características que permitan la estabilidad del sistema. En particular, hacemos uso del valor de Shapley \cite{shapley} como concepto de solución de este problema.

El valor de Shapley proporciona una distribución justa de los costos con la ventaja de ser una respuesta única y puede representar una asignación consistente en el tiempo \cite{Petrosjan}.

Sea la dupla $(S,t)$ el estado del juego en el momento $t$. Esto significa que podemos representar con $\Gamma(S,t)$ al subjuego que comienza en el momento $t$ con un nivel $S$ de personas infectadas. Denotamos con $S^I$ la trayectoria de infección acumulada bajo cooperación total, por lo que $\Gamma(S^I,t)$ denota a un subjuego que comienza bajo una trayectora de cooperación absoluta.

Sea $K \subseteq I$ una coalición de jugadores que cooperan. Se define la \emph{función característica} de una coalición en un subjuego como el costo mínimo y se denota como $v(K,S,t)$. Por ejemplo, $v(I,S,0)$ sería la forma de denotar el costo mínimo total asignado bajo cooperación total.

El valor de Shapley del subjuego $\Gamma(S,t)$ depende de la función característica y se denota por $\varphi(v,S,t) = (\varphi_1(v,S,t),\dots,\varphi_n(v,S,t))$. Para obtener soluciones estables en el tiempo se requiere definir un proceso de distribución de asignaciones. Para lograr esto, definimos a $\beta_i(t)$ el costo asignado al jugador $i$ en el instante $t$ y definimos el vector $\beta(t) = (\beta_1(t),\dots, \beta_n(t))$.

\begin{definition}
El vector $\beta(t) = (\beta_1(t),\dots, \beta_n(t))$ es un \emph{proceso de distribución de asignaciones} (PDA) si 

\begin{equation}
	\varphi_i(v,S,0) = \int_0^\infty \eu^{-\rho t} \beta_i(t) \du t, \qquad  i = 1,\dots,n.
\end{equation}
\end{definition}

La forma de interpretar esta definición es que una función $\beta_i(t)$ constituye una PDA si esta descompone en el tiempo el costo total descontado del jugador $i$ dado por el valor de Shapley para el juego total $\Gamma(S,0)$. Esto se podría leer como que la suma de costos instantáneos descontados es igual a $\varphi_i(v,S,0)$.

\begin{definition}\label{tcIDP}
El vector $\beta(t) = (\beta_1(t),\dots, \beta_n(t))$ es un PDA consistente en el tiempo si en $(S^I, t), \forall t \in [0,\infty)$ se da la siguiente condición
\begin{equation}
	\varphi_i(v,S,0) = \int_0^t \eu^{-\rho t} \beta_i(\tau) \du \tau+ \eu^{-\rho t} \varphi(v, S^I, t)
\end{equation}
\end{definition}

Para comprender la condición presentada por la definición \ref{tcIDP}, considere cualquier momento $t$ del juego en el que los jugadores desean renegociar los acuerdos realizados al inicio dentro de la gran coalición. El estado del juego es, por lo tanto, $(S^I,t)$, y a todos los jugadores se les ha estado asignando la suma de los costos descontados correspondientes a la primera parte del lado derecho de la ecuación. En el caso de que este jugador decida jugar de manera cooperativa en el subjuego $\Gamma(S^I, t)$, entonces tendría que recibir su componente del valor de Shapley. Este se encuentra representado por la segunda parte del lado derecho de la ecuación \eqref{tcIDP}.

Encontrar un PDA $\beta(t) = (\beta_1(t), \dots, \beta_n(t))$ tal que se de la condición de la ecuación \eqref{tcIDP} implica la existencia de un acuerdo que es consistente en el tiempo. En la siguiente sección mostramos el procedimiento que usamos para la obtención del resultado.

\section{Obtención de un PDA consistente en el tiempo}
Nuestro objetivo inicial es la obtención de la función de valor del juego. Esto requiere el cálculo de los costos mínimos para cada uno de los agentes, la gran coalición y cada una de las sub-coaliciones que se pueden generar. Esto permitirá generar el valor de Shapley y calcular las funciones $\beta_i(t)$ para cada uno de los jugadores.

El primer paso es el cálculo del costo mínimo para la gran coalición. Este es el problema mas simple, pues se trata a la suma de todos los costos de los agentes como un solo problema de programación dinámica, sujeta a la dinámica de la acumulación de casos.

\begin{align}
	\min &\qquad \sum_{i\in I} W_i(e,S,t) = \sum_{i\in I} \int_t^\infty \eu^{-\rho(\tau - t)}
	\{
		C_i(e_i) + D_i(S)
	\} \du \tau \\
	\mbox{t.q. } &\qquad \dot{S}  = \sum_{i \in I} e_i - \delta S, \qquad S(t) = S^I(t)
\end{align}

Sea $V(I,S, t)$ la función de valor de Bellman del problema estándar de programación dinámica, se acuerdo al teorema XXXX en XXXX, esta debe de cumplir con la ecuación de Bellman

\begin{equation}\label{bellmanCoalition}
	\rho V(I,S,t) = \min \left\{
		\sum_{i=1}^n \left(
		\frac{\gamma}{2}[e_i - \bar{e}_i]^2 + \pi S
		\right) + V'(I,S,t) \left[
			\sum_{i=1}^n e_i(t) - \delta S(t)
		\right]
	\right\}
\end{equation}

Para encontrar la estrategia óptima que los agentes deben tomar en conjunto, es necesario diferenciar el lado derecho de la ecuación con respecto a $e_i$ igualando a cero

\begin{equation}
	e_i^I = \bar{e}_i - \frac{1}{\gamma} V'(I,S,t).
\end{equation}

Sustituyendo $e_i^I$ en \eqref{bellmanCoalition},

\begin{equation}\label{bellmanCoalition}
	\rho V(I,S,t) = \min \left\{
		\sum_{i=1}^n \left(
		\frac{\gamma}{2}[e_i - \bar{e}_i]^2 + \pi S
		\right) + V'(I,S,t) \left[
			\sum_{i=1}^n e_i(t) - \delta S(t)
		\right]
	\right\}
\end{equation}


\bibliographystyle{unsrt}
\bibliography{references}

\end{document}  